\documentstyle{article}
\begin{document}

\title{AAUW Panel Intro}
\author{Ellen Spertus}
\date{January 29, 1994}
\maketitle

\section{Introduction}

I got my Bachelor's and Master's degree and am working on my PhD in
computer science at MIT.  After seven-and-a-half years at MIT, I still
love the place, but I've also seen serious problems.  I'm going to
discuss three aspects of my MIT experience: my computer science
education, my development as a feminist, and other women's experiences
at MIT.

\section{Computer Science Education}

When I got my bachelor's degree, I wrote a letter to the department
head, Paul Penfield, from which I will read:
\begin{quotation}
I am very happy with my undergraduate experience at MIT.  I have
enjoyed being here, and I feel I have benefitted greatly.  Because of
the history of discrimination against women in Computer Science here
and elsewhere, I had been worried that, at the very least, professors
would have lower expectations of me and would not encourage me to
tackle difficult problems.  I am very happy to say this was not the
case.  All of the professors I have dealt with have been wonderful.  I
would like to call to your attention the [two] whom I have dealt with
most.

Arvind is my academic advisor.  He always treats me with respect and
encouraged me to take the most challenging classes.  He once even told
me my schedule was too easy for me and that I should add more courses.
This occurred at a time when I had been sensitized to problems other
women have had with their advisors trying to steer them into softer
courses, so I especially appreciated the treatment.  Arvind has also
expressed interest in my thesis and stayed late one evening just so
he can attend my presentation, which was flattering....

My biggest debt is to my thesis supervisor Bill Dally.  He gave me an
extremely challenging thesis topic and the resources to do it.  He
meets with each of his students --- including undergraduates --- for
an hour every two weeks.  I have not heard of any other faculty member
devoting so much energy to undergraduates.  When I wanted him to
review my thesis outline before I left town, he took unscheduled time
out of a very hectic day to discuss it with me.  He normally does not
come in on Fridays, but he told me he would come on the thesis
deadline date, a Friday, just to sign my thesis, if I needed the last
day.  He  read drafts of each of my chapters, providing
excellent feedback.  I hear that most professors don't read
undergraduate theses at all.  The fun I've had doing research this
year for him is what decided me to go for my Ph.D.  
\end{quotation}

My good experiences have continued as a graduate student.  I am
treated as a valued member of the community, and professors are happy
to give me guidance.  My thesis supervisor still, Bill Dally, has not
only told me that he thinks I'm good enough to get a top academic job
when I graduate, but he's given me thorough advice on how to market
myself in the next few years to get such a job.

\section{Feminist Development}

I changed from being anti-feminist to feminist in my last two years of
college.  I can pinpoint the exact causes of the change.  During my
junior year, a female graduate student gave me a report written a
decade earlier about the chilly climate for women in computer science
at MIT.  I had assumed that women's underrepresentation in science was
because they were stupid or lazy; now, I saw that my premise was wrong
and that women hadn't been treated equally.  Let me read you one of my
favorite quotations on this subject, made by Dorothy Zinberg:
\begin{quotation}
As the data from women's career studies and anecdotes from personal
experiences of women professionals begin to accrue, one of the
questions that arises is not `Why are there so few successful
professional women?', but rather, `How have so many been able to
survive the vicissitudes on each rung of the career
ladder?'\footnote{Kundsin, Ruth B., editor.  {\it Women and Success:
The Anatomy of Achievement}.  New York: William Morrow \& Company,
1974, page 129.} 
\end{quotation}

In the first semester of my senior year, I took a course, ``Women in
Literature'', taught by an MIT feminist (no, that's not a
contradiction in terms) Ruth Perry.  She had decided to teach the
course because when it had been offered the previous year, all of the
readings were written by men, as if men were uniquely qualified to
write about women.  In contrast, she exclusively chose books written
by women.  I was sufficiently skeptical that I expected to read
second-rate books that were only included because men's works were
excluded.  I was wrong.  I read {\it Middlemarch}, {\it The Street},
and {\it Their Eyes Were Watching God} and wanted to know {\it why
nobody ever recommended these books to me before.  They're great
books!} Only later did I realize that {\it none\/} of the books
assigned to me in high school were by women.  This must have
contributed to my falsely concluding that no first-rate books were by
women.  Being proved wrong led me to see how women's work is sometimes
undervalued.

An even more significant experience was taking Sherry Turkle's course,
``Women and Computers.''  From the readings and discussions, I learned
much more about factors that worked against women in science and
engineering.  For example, I read the ``chilly climate'' reports for
the first time.

When Prof.~Turkle announced at the beginning of the class that we
would have to write a twenty-page term paper, I asked incredulously
how we'd be able to write that much.  I ended up writing over 100
pages on reasons for women's underrepresentation in computer science.
My report was read first by MIT students and professors (including the
department head), then spread through Systers, the electronic mailing
list for women in computer science, and has spread like wildfire even
among male computer scientists --- for example, the Computer Science
head at Berkeley (one of the best CS departments) gave copies to all
the graduate students and faculty; I was one of the few MIT students
invited to testify before the President's Council of Advisors on
Science and Technology; I was invited to write an article on the
subject in the {\it Encyclopedia of Computer Science}, and My
department gave me the award for outstanding research by an
undergraduate for the report.  (The wonderful department head I keep
mentioning did goof a little: he addressed the award letter to ``{\it
Mr.}~Ellen Spertus''!)  With the department's support, I co-founded a
big sister program, and I'm a member of a committee on women's
enrollment.

One of the requirements in the graduate program is to take a minor
consisting of three classes.  I got a minor in women's studies
approved (although I switched for scheduling reasons to Science,
Technology, and Society) and have been fortunate enough to take two
classes from Evelyn Fox Keller, well-known for her work on gender and
science.  I took another class from Mike Fischer, a respected
anthropologist, in which I learned how to study scientists and
engineers, something I'd been doing informally for years.

This month, I attended a wonderful unique conference: ``Black Women in
the Academy: Defending Our Name.''  There were keynote addresses by
Lani Guinier, Johnetta Cole (the president of Spelman College), and
Angela Davis and panels on such as the role of women as Tuskegee
Institute.  There definitely {\it is\/} feminism at MIT.

A priori, I feared that I would have to pay a price for being a
feminist, with people being mean to me or assuming that my computer
science research was second-rate.  This has not happened.  I was
pleasantly surprised how many of the male students and professors are
interested in these issues.  Outside of the department, the report has
made me much more visible than I would have been otherwise, and people
are interested in finding out about my computer science research.
While I'm sure there are departments that would consider me
undesirable for being such a vocal feminist, I think this is the
exception rather than the rule.

\section{Other Women's Experiences}

I wish I could say that other women's MIT experiences have been as
good as mine.  I have heard about bias in mathematics and physics
especially.  I have been told by multiple sources that there is a
physics professor who tells female undergraduates that they don't
belong at MIT and only got in through affirmative action.  Not only is
the statement unkind, but it is untrue.  Women's grade point averages
are the same as men's, even when adjusted for major, and they graduate
at a higher rate.  In Computer Science, women's GPAs are higher than
men's, but most undergraduates think the opposite is true.  One
undergraduate woman told me that her boyfriend said she'd do badly in
the introductory Computer Science course.  Not only did she get an A,
she became a lab assistant for the course (and is no longer dating
that boyfriend).  Other women have complained about the lower
expectations professors and fellow students seem to have for them.

I know several women who have received far too much romantic or sexual
attention from male colleagues, up to several cases bordering on
stalking, and a reported attempted rape (reported to me, not any
authorities).  I once had to call hotel security when a roommate of
mine was stalked at her first computer science conference.  Many men
(and some women) are unaware that this sort of thing goes on.

\section{Conclusions}

As for conclusions, the simplest one I can draw is that a woman can
avoid harassment and bias if her technical credentials are impeccable,
she doesn't attract attention through being conspicuously beautiful or
ugly, she's assertive, and she's lucky enough to fall in with
professors and students who treat women (and people in general) well.
There is an awareness among many at MIT that more needs to be done to
make the environment unbiased against women, although there are others
who think too much has been done, such as that a physics professor is
not permitted to have a belly dancer perform before the class to
illustrate ``waves.''  Although some women, such as myself, have
had excellent experiences, much remains to be done.

\end{document}


